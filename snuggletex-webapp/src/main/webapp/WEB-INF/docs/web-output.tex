\pageId{webOutput}

\section*{Creating Web Pages}

SnuggleTeX tries to make it easy to create web pages suitable for use
on various browser platforms.

\subsection*{Usage}

There are two main groups of methods for generating fully-blown web pages:

\begin{itemize}

  \item The \verb|snuggleSession.createWebPage()| methods return a DOM \verb|Document|
    representing a web page rendition of your input. You can then serialize this or
    perform further transforms/manipulations as required.

  \item The \verb|snuggleSession.writeWebPage()| methods create and write out a web page
    rendition of your input to the given output destination.

\end{itemize}

These methods take a \verb|WebPageOutputOptions| Object which specifies exactly
what you want to do.

The \verb|MathMLWebPageOptions| Object extends \href[\verb|DOMOutputOptions|]{dom-output.html},
giving you additional control over the following aspects of the output:

\begin{itemize}
  \item Which ``Page Type'' to use for the resulting web page;
  \item Set a language, encoding and title for the resulting page;
  \item Specify client-side XSLT stylesheets to include via \verb|<?xml-stylesheet?>|;
  \item Specify client-side CSS stylesheets to link to via \verb|<link/>|;
  \item Whether to add an automatic title heading element to the web page body;
  \item Whether to indent the output;
  \item Whether to apply your own XSLT Stylesheets (passed as JAXP \verb|Transformer| Objects)
    to the resulting page before it is serialized. (This is useful if you want to add
    in custom headers and footers or otherwise soup up the outputs you get.)
\end{itemize}

Because generating and serving MathML can be difficult and error-prone, SnuggleTeX provides
a \verb|WebPageOutputOptionsTemplate.createWebPageOptions()| helper method to
help you create suitable instances of \verb|WebPageOutputOptions| from a given
\verb|WebPageType| that you can use as-is or tweak as required.

\subsection*{Supported Page Types}

\begin{itemize}
  \item \anchor{mozilla} \textbf{MOZILLA} \href[(example)]{math-mode.xhtml}:
    This generates an XHTML + MathML document suitable for
    Mozilla-based browsers (e.g. Firefox). It is served as \verb|application/xhtml.html|,
    with no XML declaration and no DOCTYPE declaration. Do \emph{not} use this if you
    are targeting Internet Explorer as it will display this as an XML tree.

  \item \anchor{crossbrowser} \textbf{CROSS\_BROWSER\_XHTML} \href[(example)]{math-mode.cxml}:
    This generates an XHTML + MathML document that
    displays well on both Mozilla-based browsers and Internet Explorer 6 and above (providing that
    the MathPlayer plug-in has already been installed).
    It is served as \verb|application/xhtml.html|,
    has an XML declaration and a DOCTYPE declaration.
    \begin{itemize}
      \item This will not display correctly on Internet Explorer if MathPlayer has not
        already been installed on it by your target user.
      \item Internet Explorer will download the DTD, which will slow rendering down
        somewhat.
    \end{itemize}

  \item \anchor{mathplayer} \textbf{MATHPLAYER\_HTML} \href[(example)]{math-mode.html}:
    This generates an HTML document that displays well
    on Internet Explorer 6 and above with the MathPlayer plugin installed. It will not display
    MathML correctly on Mozilla-based browsers.

  \item \anchor{uss} \textbf{UNIVERSAL\_STYLESHEET} \href[(example)]{math-mode.xml}:
    This generates an XHTML + MathML document
    that is served as XML and can be served to both Mozilla-based browsers and
    Internet Explorer 6+. It includes an XML processing instruction that tells
    browsers to apply the Universal MathML StyleSheet to the page before delivering,
    prompting IE users to download and install MathPlayer if they do not already have it.
    \begin{itemize}
      \item IE requires that client-side XSLT stylesheets are loaded from the same server
        that the document came from, so you must copy the USS bundled with SnuggleTeX to
        your own server and tell SnuggleTeX where you put it by calling the
        \verb|setClientSideXSLTStylesheetURLs()| method.
      \item This can be a very good option for decent portability, though it does slow
        rendering down somewhat and may also increase the load on your server as some
        browsers will load static resources both before and after the XSLT is applied.
    \end{itemize}

  \item \anchor{xsl} \textbf{CLIENT\_SIDE\_XSLT\_STYLESHEET}:
    This generates a XHTML + MathML document,
    served as \verb|application/xhtml.html| with no XML declaration and no DOCTYPE declaration.
    It is intended to be used with a client-side XSLT of your choice.
    (This is a more advanced option!)

  \item \textbf{PROCESSED\_HTML}:
    This is another advanced option that assumes it will generate a plain old
    HTML document, served as \verb|text/html|.
    You will have to post-process the resulting DOM to produce something sensible here
    by registering a \verb|DOMPostProcessor| with your \verb|WebPageOutputOptions|.
    This feels like a rather esoteric option, but is actually where the SnuggleTeX JEuclid
    module hooks in when converting the results to legacy HTML + Images.

\end{itemize}

\subsection*{\anchor{legacy}Generating Legacy HTML + Images Web Pages}

The full distribution of SnuggleTeX includes a module called \verb|snuggletex-jeuclid.jar|
that uses the JEuclid library to convert MathML to images, optionally attempting to
convert simpler expressions to XHTML+CSS as well.

The \verb|JEuclidUtilities.createWebPageOptions()| method created a suitably
configured instance of \verb|WebPageOutputOptions| for this. (The resulting
\verb|WebPageType| in this case is \verb|PROCESSED_HTML|, but you don't have to
worry about this.)
