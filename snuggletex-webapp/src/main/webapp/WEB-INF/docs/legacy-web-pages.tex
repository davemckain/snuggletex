\pageId{legacyWebPages}

In some scenarios, it might not be viable to use the MathML-based web pages that
SnuggleTeX generates by default. However, all is not lost! The full distribution
of SnuggleTeX ships with an extension that converts MathML to images using components
of the \href[JEuclid]{http://jeuclid.sourceforge.net/} software.

(SnuggleTeX can also attempt to convert very simple MathML expressions into XHTML+CSS
using the \verb|DownConvertingPostProcessor| class that you can easily attach to your
\verb|WebPageOutputOptions|.)

Using this extension requires a bit of programming on your part as you must decide
how you are going to store and serve the resulting images along with your web pages.

\subsection*{Getting Started with the JEuclid Extension}

\begin{itemize}

\item
  In order to use the JEuclid Extension, you will have to download the
  full distribution of SnuggleTeX.

\item
  You need to have \verb|snuggletex-core.jar| and
  \verb|snuggletex-jeuclid.jar| in your ClassPath, as well as
  \verb|jeuclid-core.jar|,
  \verb|batik-dom.jar|,
  \verb|batik-ext.jar|,
  \verb|batik-util.jar|,
  \verb|batik-xml.jar|,
  \verb|commons-iojar|,
  \verb|commons-logging.jar|,
  \verb|xml-apis.jar (not needed for Java 6 and above)|,
  \verb|xml-apis-ext.jar| and
  \verb|xmlgraphics-commons.jar|.
  (The JARs in the distributon
  ZIP file contain version numbers in their names; it should be obvious what
  we mean here.)

\end{itemize}

\subsection*{Usage}

The \verb|JEuclidUtilities.createWebPageOptions()| method creates a suitably
configured instance of \verb|WebPageOutputOptions| that will convert MathML to images
(trying XHTML+CSS first if requested) and produce a legacy HTML web page as a result.

In order to make this work correctly, you need to pass this method an instance of
an interface called \verb|MathMLImageSavingCallback| that tells SnuggleTeX what to do
with each image it creates. (There is an abstract implementation of this called
\verb|SimpleMathMLImageSavingCallback| that you might find a useful starting point here.)

You can then use this \verb|WebPageOutputOptions| to generate SnuggleTeX web page
outputs in the usual way.

\subsection*{Technical Implementation Notes}

The resulting \verb|WebPageType| in this case is actually \verb|PROCESSED_HTML|,
which has been configured with one or both of the following \verb|DOMPostProcessor|s:

\begin{itemize}
\item \verb|DownConvertingPostProcessor|: This will optionally attempt to convert
simple MathML expressions to XHTML + CSS.

\item \verb|JEuclidMathMLPostProcessor|: This hunts out any (remaining) MathML elements,
converts them to images using JEuclid, calls back on your \verb|MathMLImageSavingCallback|
to let you save them or store them, and then replaces the MathML with XHTML
\verb|<img/>| elements that you control using your callback.

\end{itemize}
