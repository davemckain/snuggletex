\pageId{textMode}

% Simple input/result table environment
\newenvironment{demotable}
{\begin{center}
 \begin{tabular}{|l|l|}
 \hline \\
 Input & Result \\
 \hline \\
}{\hline
 \end{tabular}
 \end{center}
}

\newenvironment{ndemotable}
{\begin{center}
 \begin{tabular}{|l|l|l|}
 \hline \\
 Input & Result & Notes \\
 \hline \\
}{\hline
 \end{tabular}
 \end{center}
}

% Text-mode demo line
\newcommand{\note}[1]{\small #1}
\newcommand{\demo}[1]{\verb|#1| & #1 \\}
\newcommand{\ndemo}[2]{\verb|#1| & #1 & \note{#2} \\}
\newcommand{\bigdemo}[1]{\begin{verbatim}#1\end{verbatim} & #1 \\}

\subsection*{Text Escapes}

SnuggleTeX supports the usual commands for escaping characters that are normally
special in LaTeX.

\begin{ndemotable}
\demo{\$}
\demo{\%}
\demo{\#}
\demo{\&}
\demo{\_}
\demo{\{}
\demo{\}}
\ndemo{No\ break}{Produces a non-breaking space}
\end{ndemotable}

\subsection*{Text Spacing Commands}

\begin{demotable}
\demo{A\,B}
\demo{A\hspace{2em}B}
\end{demotable}

\subsection*{Punctuation}

\begin{ndemotable}
\demo{`Single Quotes'}
\demo{``Double Quotes''}
\ndemo{Yes --- but no!}{Dash}
\ndemo{1--2}{Dash used for numerical ranges}
\ndemo{Mr.\ Blitherington-Smythe}{Single '-' is a hyphen}
\ndemo{No~Break}{Non-breaking space}
\demo{Boring\ldots}
\demo{<}
\demo{>}
\demo{|}
\end{ndemotable}

\subsection*{Lists}

SnuggleTeX supports the {\tt itemize} and {\tt enumerate} environments.

\begin{demotable}
\bigdemo{\begin{itemize}
  \item First item
  \item Second item
\end{itemize}}
\bigdemo{\begin{enumerate}
  \item First item
  \item Second item
\end{enumerate}}
\end{demotable}

\subsection*{Text Styling}

These commands style text using the old TeX approach,
causing the new style to be applied until the end of
the current level (i.e. closing curly bracket).

\newcommand{\styledemo}[1]{\demo{#1 Some Text}}
\begin{demotable}
\styledemo{\em}
\styledemo{\bf}
\styledemo{\rm}
\styledemo{\it}
\styledemo{\tt}
\styledemo{\sc}
\styledemo{\sl}
\styledemo{\sf}
\end{demotable}

\subsection*{Newer Text Stylings}

The following commands are similar but the styles apply to their
arguments explicitly.

\newcommand{\textstyledemo}[1]{\demo{#1{Some Text}}}

\begin{demotable}
\textstyledemo{\textrm}
\textstyledemo{\textsf}
\textstyledemo{\textit}
\textstyledemo{\textsl}
\textstyledemo{\textsc}
\textstyledemo{\textbf}
\textstyledemo{\texttt}
\textstyledemo{\emph}
\end{demotable}

\subsection*{Sizing Commands}

The standard TeX sizing commands are supported. These apply until the
end of the current level (i.e. closing curly bracket).

\begin{demotable}
\styledemo{\tiny}
\styledemo{\scriptsize}
\styledemo{\footnotesize}
\styledemo{\small}
\styledemo{\normalsize}
\styledemo{\large}
\styledemo{\Large}
\styledemo{\LARGE}
\styledemo{\huge}
\styledemo{\Huge}
\end{demotable}

\subsection*{Text Mode Accents}

Applying accents to characters in XHTML is more difficult to implement than
you might except as it usually requires remapping Unicode characters to their
accented equivalents, requiring appropriate fonts to be installed to ensure
they display correctly. As a result, SnuggleTeX currently only supports a
subset of the accenting that you would normally expect; it will report
error \href[\verb|TDETA2|]{error-codes.html#TDETA2} if can't create a particular accent you requested.
We may add support for less common accented characters provided the results
work well across all target browsers and platforms. (You might be better
creating Math-mode accents in these cases.)

The table below shows the characters that SnuggleTeX currently accents correctly.
The exception here is \verb|\underline|, which SnuggleTeX implements using CSS
so can safely be applied to any character (provided of course you ensure that
the CSS is correctly used).

\begin{demotable}
\demo{\'a\'e\'i\'o\'u\'y\'A\'E\'I\'O\'U}
\demo{\`a\`e\`i\`o\`u\`A\`E\`I\`O\`U}
\demo{\^a\^e\^i\^o\^u\^A\^E\^I\^O\^U}
\demo{\~a\~n\~o\~A\~O}
\demo{\"a\"e\"i\"o\"u\"A\"E\"I\"O\"U}
\demo{\underline{a} et.\ al.}
\end{demotable}

\subsection*{Miscellaneous Text-mode Symbols}

We also support the following basic text-mode symbols:

\begin{demotable}
\demo{\ae}
\demo{\oe}
\demo{\dag}
\demo{\pounds}
\end{demotable}


