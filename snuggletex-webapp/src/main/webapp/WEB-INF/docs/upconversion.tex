\pageId{browserRequirements}

\section*{MathML Semantic Up-Conversion}

To write about...

\begin{itemize}
\item Failure handling
\item Conventions followed
\end{itemize}

\subsection*{Presentation MathML Enhancement}

The first part of the up-conversion process takes the rather flat MathML normally
output by SnuggleTeX and creates ``enhanced'' presentation MathML that displays in the
same way whilst having a structure that is more amenable to further processing,
such as conversion to Content MathML and other forms.

Generally speaking, the improvements made are as follows:

\begin{itemize}
\item Precedence of infix operators is inferred, as described below, using
\verb|<mrow/>| to house inferred arguments.
\item Implicit multiplications are inferred, using \verb|<mo>&InvisibleTimes;</mo>|
to represent this in the resulting MathML.
\item Applications of known functions like \verb|sin| are inferred, using
\verb|<mo>&ApplyFunction;</mo>| to represent this.
\end{itemize}

\textbf{TODO:} Talk about conventions followed.

\begin{tabular}{|c|l|}
\hline
Test & Result \\
\hline
Infix $,$ & Grouped into a \verb|<mfenced>| with empty opener and closer \\
Infix $\vee$ & Associative Grouping \\
Infix $\wedge$ & Associative Grouping \\
Infix relation operator(s) & Associative Grouping (all at same level) \\
Infix $\cup$ & Associative Grouping \\
Infix $\cap$ & Associative Grouping \\
Infix $\setminus$ & Left-associative Grouping \\
Infix $+$ & Associative Grouping \\
Infix $-$ & Left-associative Grouping \\
Infix $*$, $\times$ and $\cdot$ & Associative Grouping (all at same level) \\
Infix $/$ and $\div$ & Left-associative Grouping \\
Space characters & Treated as explicit multiplication, grouped associatively \\
Any infix operator in unary context & Operator ``appplied'' by wrapping in \verb|<mrow/>| \\
No Infix Operator present & Apply into subgroups (as defined below) and apply implicit product \\
Atom & Kept as-is \\
\hline
\end{tabular}

\subsubsection*{No Infix Operator Handling}

This is split into subgroups starting with elements satisfying
any of the following:

\begin{itemize}
\item The first sibling in a group
\item The first sibling following an \verb|<mfenced/>|
\item The first of one or more prefix operator or function siblings
\item The first non-postfix operator after one or more postfix operator siblings
\end{itemize}

For example, the LaTeX source:

\begin{verbatim}
xy \sin 2x \sin\cos 2x!
\end{verbatim}

would identify 3 subgroups: \verb|xy|, \verb|\sin 2x| and \verb|\sin\cos 2x!|.

The rules for handling these subgroups are listed below. The results of these
are turned into an implicit product using \verb|<mo>&InvisibleTimes;</mo>|.

\subsubsection*{Subgroup Handling}

\begin{itemize}
\item Firstly, any prefix functions (e.g. \verb|\sin|) and operators (e.g. \verb|\lnot|)
are applied recursively from left to right to everything following.

So, \verb|\sin\cos 2ax!| is handled as it if were \verb|\sin(\cos(2ax!))|.

\item Once all prefix functions and operators have been applied, any postfix operators
attached to whatever is remaining are applied from right to left. (Note however
that the factorial operator is handled specially so that it only gets applied
to the last token, so that \verb|2ax!| is treated as \verb|2a(x!)|, which fits in
with common conventions.)

\item Finally, everything between the prefix and postfix parts are treated as an
implicit multiplication.
\end{itemize}

For example, the result of this work on \verb|\sin\cos 2ax!| is the equivalent of
\verb|\sin(\cos(2 \cdot a \cdot x!))|.

\subsection*{Conversion to Content MathML}

The second step in the up-conversion process takes the enhanced Presentation MathML
and attempts to convert it into Content MathML.

\subsubsection*{Configurable Assumptions}

\begin{itemize}
\item Whether $e$ should be treated as the exponential number or not.
\item Whether $i$ should be treated as the imaginary number.
\item Whether $\pi$ should be treated as the number $3.141592\ldots$.
\item Whether $(\ldots)$ should be assumed to delimit a vector or not.
\item Whether $[\ldots]$ should be assumed to delimit a list or not.
\item Whether $\{\ldots\}$ should be assumed to delimit a set or not.
\end{itemize}

\subsubsection*{Supported Operators}

\begin{tabular}{|c|c|c|}
\hline
LaTeX operator & Application & Content MathML element \\
\hline
\verb|\vee| & n-ary & \verb|or| \\
\verb|\wedge| & n-ary & \verb|and| \\
Any mix of relation operators & binary, applied in adjacent pairs & See below \\
\verb|\cup| & n-ary & \verb|union| \\
\verb|\cap| & n-ary & \verb|intersect| \\
\verb|\setminus| & binary & \verb|setdiff| \\
\verb|+| & unary or n-ary & \verb|plus| \\
\verb|-| & unary or binary & \verb|minus| \\
Any multiplication & n-ary & \verb|times| \\
Any division & binary & \verb|times| \\
\verb|!| & unary & \verb|factorial| \\
\hline
\end{tabular}

\begin{itemize}
\item
Operators may be left ``unapplied'', e.g. a raw input of \verb|+|
would result in \verb|<plus/>| with no enclosing
\verb|<apply/>|.

\item
Failures will be registered if an operator is used in an inappropriate
context.
\end{itemize}

\subsubsection*{Supported Relation Operators}

\begin{tabular}{|c|c|}
\hline
LaTeX operator & Content MathML element \\
\hline
\verb|=| & \verb|<eq/>| \\
\verb|\not=| & \verb|<neq/>| \\
\verb|<| & \verb|<lt/>| \\
\verb|\not<| & \verb|<not>...<lt/>...</not>| \\
\verb|>| & \verb|<gt/>| \\
\verb|\not>| & \verb|<not>...<gt/>...</not>| \\
\verb|\leq| & \verb|<leq/>| \\
\verb|\not\leq| & \verb|<not>...<leq/>...</not>| \\
\verb|\geq| & \verb|<geq/>| \\
\verb|\not\geq| & \verb|<not>...<geq/>...</not>| \\
\verb|\equiv| & \verb|<equivalent/>| \\
\verb|\not\equiv| & \verb|<not>...<equivalent/>...</not>| \\
\verb|\approx| & \verb|<approx/>| \\
\verb|\not\approx| & \verb|<not>...<approx/>...</not>| \\
\verb.|. & \verb|<factorof/>| \\
\verb.\not|. & \verb|<not>...<factorof/>...</not>| \\
\verb|\in| & \verb|<in/>| \\
\verb|\not\in| & \verb|<notin/>| \\
\hline
\end{tabular}

\subsubsection*{Supported Functions}

\begin{tabular}{|c|c|c|c|}
\hline
LaTeX function & Arity & Invertible & Content MathML element \\
\hline
\verb|\sin| & unary & Yes & \verb|sin| or \verb|arcsin| \\
\verb|\min| & n-ary & No & \verb|min| \\
$\ldots$ & & & \\
\hline
\end{tabular}

