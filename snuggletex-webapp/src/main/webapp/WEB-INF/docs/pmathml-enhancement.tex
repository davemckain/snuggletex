\pageId{pmathmlEnhancement}

\newcommand{\ue}[1]{\upConversionExample{#1}}

The first part of the up-conversion process takes the rather flat MathML normally
output by SnuggleTeX and creates ``enhanced'' presentation MathML that displays in the
same way whilst having a structure that is more amenable to further up-conversion,
such as the conversion to Content MathML and Maxima input syntax.

Generally speaking, the enhancements made are as follows:

\begin{itemize}
\item Precedence of infix operators is inferred, as described in the table 
below, using \verb|<mrow/>| to encapsulate inferred arguments.
\item Implicit multiplications are inferred, using \verb|<mo>&InvisibleTimes;</mo>|
to represent this in the resulting MathML.
\item Applications of supported functions like \verb|sin| are inferred, using
\verb|<mo>&ApplyFunction;</mo>| to represent this.
\end{itemize}

\subsection*{Precedence Table}

The enhancement process works on a set of MathML siblings by applying
each test in the table below, in the order shown. When the first match
occurs, the siblings are converted into a new form and the process then
descends downwards.

\begin{tabular}{|c|c|c|}
\hline
Test & Result & Example \\
\hline
Infix $,$ & Grouped into a \verb|<mfenced>| with empty opener and closer & \ue{\verb|x,y,z+1|} \\
Infix $\vee$ & Associative Grouping & \ue{\verb|x\vee \lnot y|} \\
Infix $\wedge$ & Associative Grouping & \ue{\verb|x\vee y \wedge z|} \\
Infix relation operator(s) & Associative Grouping (all at same level) & \ue{\verb|1\leq x-a < 2|} \\
Infix $\cup$ & Associative Grouping & \ue{\verb|A\cup B \cap C|} \\
Infix $\cap$ & Associative Grouping & \ue{\verb|A\cup B \cap C|} \\
Infix $\setminus$ & Left-associative Grouping & \ue{\verb|A\setminus B+x|} \\
Infix $+$ & Associative Grouping & \ue{\verb|x-1+y-2|} \\
Infix $-$ & Left-associative Grouping & \ue{\verb|--x-y-z|} \\
Infix $*$, $\times$ and $\cdot$ & Associative Grouping (all at same level) & \ue{\verb|2x+5\times (y-4)|} \\
Infix $/$ and $\div$ & Left-associative Grouping & \ue{\verb|a/b/c/(1 \div x)|} \\
Space characters & Treated as explicit multiplication, grouped associatively & \ue{\verb|a\,b|} \\
Any infix operator in unary context & Operator ``appplied'' by wrapping in \verb|<mrow/>| & \ue{\verb|-+x|} \\
No Infix Operator present & Apply into subgroups (as defined below) and apply implicit product & \ue{\verb|\sin x\cos y|} \\
Atom & Kept as-is & \ue{\verb|\sqrt{x}|} \\
\hline
\end{tabular}

\subsection*{``No Infix Operator'' Handling}

Groups of siblings that do not contain any infix operators are treated as an
implicit product of sibling subgroups starting with elements satisfying any of
the following conditions:

\begin{itemize}
\item The first sibling
\item The first sibling following an \verb|<mfenced/>|
\item The first of one or more prefix operator or function siblings
\item The first non-postfix operator after one or more postfix operator siblings
\end{itemize}

For example, the LaTeX source:

\begin{verbatim}
xy \sin 2x \sin\cos 2ax!
\end{verbatim}

would identify 3 subgroups: \verb|xy|, \verb|\sin 2x| and \verb|\sin\cos 2ax!|.

The rules for handling these subgroups are listed below. The results of these
are turned into an implicit product using \verb|<mo>&InvisibleTimes;</mo>|.

\subsection*{Sibling Subgroup Handling}

\begin{itemize}
\item Firstly, any prefix functions (e.g. \verb|\sin|) and operators (e.g. \verb|\lnot|)
are applied recursively from left to right to everything following.

So, \verb|\sin\cos 2ax!| is handled as it if were \verb|\sin(\cos(2ax!))|.

\item Once all prefix functions and operators have been applied, any postfix operators
attached to whatever is remaining are applied from right to left. (Note however
that the factorial operator is handled specially so that it only gets applied
to the last token, so that \verb|2ax!| is treated as \verb|2a(x!)|, which fits in
with common conventions.)

\item Finally, everything between the prefix and postfix parts are treated as an
implicit multiplication.
\end{itemize}

For example, the result of this work on \verb|\sin\cos 2ax!| is the equivalent of
\verb|\sin(\cos(2 \cdot a \cdot x!))|.
