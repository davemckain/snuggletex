\pageId{pmathmlEnhancement}

\newcommand{\ue}[1]{\upConversionExample{#1}}

The first part of the up-conversion process takes the rather flat MathML normally
output by SnuggleTeX and creates ``enhanced'' presentation MathML that displays in the
same way whilst having a structure that is more amenable to further up-conversion,
such as the conversion to Content MathML and Maxima input syntax.

Generally speaking, the enhancements made are as follows:

\begin{itemize}
\item Precedence of infix operators is inferred, as described in the table
below, using \verb|<mrow>...</mrow>| to make groupings and delimit inferred arguments.
\item Implicit multiplications are inferred, using \verb|<mo>&InvisibleTimes;</mo>|
(a.k.a.\ \verb|<mo>&#8290;</mo>| or \verb|<mo>&#x2062;</mo>|)
to represent this in the resulting MathML.
\item Applications of pre-defined functions like \verb|sin| and explicitly assumed
functions are inferred, using
\verb|<mo>&ApplyFunction;</mo>| (a.k.a.\ \verb|<mo>&#8289;</mo>| or \verb|<mo>&#x2061;</mo>|)
to represent this.
\end{itemize}

\subsection*{Precedence Table}

The enhancement process works on a list of one or more adjacent MathML element siblings by applying
each test in the table below, in the order shown. When the first match occurs,
the siblings are mapped into a new form, with the process being repeated in each of the operands.

\begin{tabular}{|c|c|c|}
\hline
Test & Result & Live Example \\
\hline
Infix $,$ & Grouped into a \verb|<mfenced>| with empty opener and closer and a child for each operand & \ue{\verb|x,y,z+1|} \\
Infix $\vee$ & Associative Grouping & \ue{\verb|x\vee \lnot y|} \\
Infix $\wedge$ & Associative Grouping & \ue{\verb|x\vee y \wedge z|} \\
Infix relation operator(s) & Associative Grouping (all at same level) & \ue{\verb|1\leq x-a < 2|} \\
Infix $\cup$ & Associative Grouping & \ue{\verb|A\cup B \cap C|} \\
Infix $\cap$ & Associative Grouping & \ue{\verb|A\cup B \cap C|} \\
Infix $\setminus$ & Left-associative Grouping & \ue{\verb|A\setminus B+x|} \\
Infix $+$ & Associative Grouping & \ue{\verb|x-1+y-2|} \\
Infix $-$ & Left-associative Grouping & \ue{\verb|--x-y-z|} \\
Infix $*$, $\times$ and $\cdot$ & Associative Grouping (all at same level) & \ue{\verb|2x+5\times (y-4)|} \\
Infix $/$ and $\div$ & Left-associative Grouping & \ue{\verb|a/b/c/(1 \div x)|} \\
``Space'' operators (i.e. anything producing \verb|<mspace/>|) & Treated as explicit multiplication, grouped associatively & \ue{\verb|a\,b|} \\
No Infix Operator present & Split into subgroups (as defined below) and apply implicit product & \ue{\verb|\sin x\cos y|} \\
``Atoms'' & Kept, applying conversion process to children & \ue{\verb|\sqrt{x}|} \\
\hline
\end{tabular}

\subsection*{``No Infix Operator'' Handling}

Groups of two or more MathML siblings elements that do not contain any infix
operators are treated as an implicit product of adjacent sibling subgroups
starting with MathML elements satisfying any of the following conditions:

\begin{itemize}
\item The first sibling
\item The first sibling after an \verb|<mfenced/>|
\item The first of one or more prefix operator or function siblings
\item The first non-postfix operator after one or more postfix operator siblings
\end{itemize}

The handling of these subgroups is described below. The following examples
hopefully illuminate this process in more detail:

\subsubsection*{Grouping Examples}

\begin{tabular}{|c|c|c|}
\hline
Input & Subgroups & Live Example \\
\hline
\verb|xy| & Kept together as one subgroup & \ue{\verb|xy|} \\
\verb|\sin 2x\cos y| & Split into \verb|\sin 2x| and \verb|\cos y| & \ue{\verb|\sin 2x\cos y|} \\
\verb|\sin f(x)| & Treated as \verb|\sin (f(x))|. (\verb|f| is assumed to be a function in these examples; this is configurable) & \ue{\verb|\sin f(x)|} \\
\verb|\min(x,y)z| & Split into \verb|\min(x,y)| and \verb|z|. This demonstrates the rule for handling fences and is appears reasonable here & \ue{\verb|\min(x,y)z|} \\
\verb|\sin(x+1)z| & Split into \verb|\sin(x+1)| and \verb|z|. This is perhaps contentious, but allows brackets to be used to explicitly delimit function arguments & \ue{\verb|\sin(x+1)z|} \\
\verb|x!y!| & Split into \verb|x!| and \verb|y!| & \ue{\verb|x!y!|} \\
\verb|\cos x!y!| & Split into \verb|cos x!| and \verb|y!| & \ue{\verb|\cos x!y!|} \\
\verb|\sin\cos x| & Kept together & \ue{\verb|\sin\cos x|} \\
\verb|xy\sin\cos 2ax!y!\min(x,y)a| & Split into \verb|xy|, \verb|\sin\cos 2ax!|, \verb|y!|, \verb|\min(x,y)| and \verb|a| & \ue{\verb|xy\sin\cos 2ax!y!\min(x,y)a|} \\
\hline
\end{tabular}

\subsection*{Sibling Subgroup Handling}

Once subgroups have been identified (as described above), each of these
subgroups is then split into:

\begin{itemize}
\item Zero or more prefix operators, unary/n-ary functions or infix operators applied in a unary context
\item Zero or more adjacent ``atoms''
\item Zero or more postfix operators
\end{itemize}

These are then treated as follows:

\begin{itemize}

\item Any postfix operators are ``applied'' from right to left to the
the atoms before them, using a \verb|<mo>&ApplyFunction;</mo>| to represent the
operator applications.

Note that the factorial operator is handled specially in that it only
gets applied to the preceding item, so that \verb|2ax!| is treated as
\verb|2a(x!)|, which fits in with common conventions.)

\ue{\verb|x!!|}

\item The atoms resulting after applying postfix operators are treated as an
implicit multiplication.

Revisiting the \verb|2ax!| example, we end up with a product of \verb|2|,
\verb|a| and \verb|x!|.

\ue{\verb|2ax!|}

\item Any prefix functions (e.g. \verb|\sin|) and operators
(e.g. \verb|\lnot|) are ``applied'' from left to right to
whatever is left, using a \verb|<mo>&ApplyFunction;</mo>| to represent the
function applications.

So, \verb|\sin\cos 2ax!| is handled as it if were \verb|\sin(\cos(2ax!))|.

\ue{\verb|\sin\cos 2ax!|}

\end{itemize}
