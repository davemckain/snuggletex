\pageId{migration110to120}

\subsection*{\verb|snuggletex-core| module}

There have been a small number of necessary API changes made in this module,
though they are fairly small and most users will not be affected by this.

\begin{itemize}

\item The \verb|DefinitionMap| class has been refitted into the newer and more useful
\verb|SnugglePackage|, which also allows you to register your own custom \verb|ErrorCode|s
and provide a standard Java \verb|ResourceBundle| for formatting error messages. You add
a \verb|SnugglePackage| to your \verb|SnuggleEngine| with the new \verb|addPackage()| method,
which replaces \verb|registerDefinitions()|.

\item The \verb|ErrorCode| enumeration has been replaced by an interface, and two
implementations. The first implementation \verb|CoreErrorCode| defines all of the error
codes for the \verb|snuggletex-core| module. The second defines all of the error messages
for the \verb|snuggletex-upconversion| module. This all fits into the new \verb|SnugglePackage|
model and makes error code registration much more modular.

\item There is a new \verb|ErrorGroup| interface for grouping \verb|ErrorCode|s into logic groups.
This is used mainly for documentation purposes. If you have been creating your own
\verb|ErrorCode|s, then you will need to fit this in.

\item The \verb|DOMOutputOptions| class has been tidied up by adding in a subclass called
\verb|XMLStringOutputOptions| that now houses properties relating to XML String outputs,
with a few new features such as the ability to output named entities for mathematical symbols
instead of numeric character references.

\item The \verb|SerializationMethod| inner enumeration class is now a top level enumeration
class.

\item The \verb|getDefaultDOMOptions| and \verb|setDefaultDOMOptions| methods in
\verb|SnuggleEngine| have been renamed as \verb|getDefaultDOMOutputOptions|
and \verb|setDefaultDOMOutputOptions| for consistency. The old method named will
still work for the time being but are deprecated for removal in 1.3.0.

\item \verb|SnuggleSession| now includes methods for generating XML String outputs that take
a \verb|XMLStringOutputOptions|. Older methods that took a \verb|DOMOutputOptions| and a number
of additional parameters have been marked as deprecated and will be removed in
1.3.0. Please update your code to use this (nicer) new approach.

\item The \verb|addingMathAnnotations| property of \verb|DOMOutputOptions| has been renamed
as \verb|addingMathSourceAnnotations|, which is more expliclt. Accessors for the old version
of this property have been kept but marked as deprecated and will be removed for the 1.3.0 release.
Please update your code accordingly.

\item The \verb|SNUGGLETEX_MATHML_ANNOTATION_ENCODING| constant in \verb|SnuggleConstants|
has been renamed as \verb|SNUGGLETEX_MATHML_SOURCE_ANNOTATION_ENCODING|. The old constant
has been kept around but marked as deprecated and will be removed in the 1.3.0 release.

\item The \verb|MathMLUtilities| utility class has been improved with serialization methods
the allow you to pass a \verb|SerializationOptions| instance specifying exactly how you want
the results to be serialized. Some of the existing overloaded methods doing much the same thing
have been deprecated as a result, and will be removed in 1.3.0. Use the new methods when possible!


\item Moved W3C-related constants into a \verb|W3CConstants| interface. Their old locations in
the internal \verb|Globals| class have been marked as deprecated and will be removed for the 1.3.0 release.

\item The \verb|Token| class representing a parsed LaTeX token may now have more than one
\verb|Interpretation| associated with it. This has allowed some of the existing classes in
the \verb|Interpretation| hierarchy to be simplified as they no longer have to rely on multiple
inheritance. This change should only affect people who have been writing their own
\verb|CommandHandler|s and \verb|EnvironmentHandler|s that hook into existing concepts.

\item The \verb|BuiltinCommandOrEnvironment| base class has changed to accommodate allowing
\verb|Token|s to have multiple \verb|Interpretation|s. This will affect you if you have been
defining your own \verb|BuiltinCommand|s or \verb|BuiltinEnvironment|s that make use of this feature.
Interpretation information is now stored in an \verb|EnumMap|.

\item The \verb|StylesheetManager| class has been refactored and improved and some of the
functionality previously provided by the internal \verb|XMLUtilities| class has been moved here.
Most users will not be using either of these.

\item Added the \verb|TransformerFactoryChooser| interface and 2 implementations (one using JAXP,
one hard-coded to use Saxon) that you may want to use or think about using should you ever use
a \verb|StylesheetManager|.

\end{itemize}

\subsection*{\verb|snuggletex-jeuclid| module}

The 1.2.0 full distribution ships with a newer version of JEuclid that fixes
problems previously noticed. Aside from a couple of API improvements, code compiled against
the 1.1.0 version of this module should work fine in 1.2.0.
\subsection*{\verb|snuggletex-upconversion| module}

This work is still experimental so the API should be considered somewhat unstable until
further notice.

\begin{itemize}

\item The old \verb|UpConversionParameters| constants are replaced by the
\verb|UpConversionOptions| and \verb|UpConversionOptionDefinitions| idea, which allows
both Java and LaTeX-based control over the process, and facilities the new "assumptions"
ideas in 1.2.0.

\item Some changes to the API of \verb|MahMLUpConverter| have been changed to accommodate
the above.

\item You now have to register the up-conversion \verb|SnugglePackage| with your \verb|SnuggleEngine|
if you want to use any of the commands added in 1.2.0. See the example classes for a guide. (As there
were no up-conversion LaTeX commands in 1.1.0, this will not be an issue!)

\end{itemize}
