\pageId{basicUsage}

\subsection*{The \verb|SnuggleEngine|}

To use SnuggleTeX, you first need to create a \verb|SnuggleEngine|:

\begin{verbatim}SnuggleEngine engine = new SnuggleEngine();\end{verbatim}

An instance of a \verb|SnuggleEngine| is thread-safe and can be used
to perform as many conversions as you like. It houses a number of defaults,
options, command and environment definitions that can be set if you want to
perform a number of very similar conversion processes.
See the \href[API Docs]{maven://apidocs/} for more information.

Despite its heavy-weight name, creating and setting up a \verb|SnuggleEngine|
is not expensive so don't worry about that!

\subsection*{The \verb|SnuggleSession|}

A \verb|SnuggleSession| represents a single SnuggleTeX conversion "job",
which typically consists of the following steps:

\begin{enumerate}
  \item Parse one or more \href[\verb|SnuggleInput|s]{docs://inputs}.
  \item Generate one or more Outputs (either \href[XML/DOM Outputs]{docs://domOutput}
  or \href[Web Pages]{docs://webOutput}).
  \item (Check for any \href[errors]{docs://errors} in the LaTeX reported
        by the above processes\ldots)
\end{enumerate}

You create a new \verb|SnuggleSession| from your \verb|SnuggleEngine| as follows:

\begin{verbatim}SnuggleSession session = engine.createSession();\end{verbatim}

This creates a fresh job using the current default settings in your
\verb|SnuggleEngine|. You can then call methods described in
\href[Inputs]{docs://inputs}, \href[Error Reporting]{docs://errors}
and \href[Creating XML/DOM Outputs]{docs://domOutput} or
\href[Creating Web Pages]{docs://webOutput} to do whatever it is you want to do.

You can discard a \verb|SnuggleSession| once it has done everything
you need. Note that a \verb|SnuggleSession| is \emph{not} thread safe.

See the \href[Examples]{docs://examples} for various examples of how
everything comes together.

