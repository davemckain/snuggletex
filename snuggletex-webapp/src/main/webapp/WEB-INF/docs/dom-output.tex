\pageId{domOutput}

The simplest outputs that SnuggleTeX can create are low-level XML/DOM-based outputs.

\subsection*{Usage}

\begin{itemize}
  \item
    Call one of the \verb|SnuggleSession.buildXMLString()| methods create an XML fragment
    (or, more formally, a ``well-formed external general parsed entity'')
    representing everything that has been parsed so far.
    This type of output can be useful for demonstrating results or if you need to pass or
    transmit the resulting XML to other pieces or software.

  \item
    Call one of the \verb|snuggleSession.buildDOMSubtree()|
    methods to append the resulting XML to an existing \verb|Element|'s
    children.  This might be useful if you are integrating SnuggleTeX into some
    other activity that builds a DOM.

  \item
    Call one of the \verb|snuggleSession.buildDOMSubtree()|
    methods to create a new \verb|Document| and return a \verb|NodeList|
    representing your converted LaTeX inputs. (Note that SnuggleTeX will wrap
    these Nodes within a special root \verb|Element| within the
    \verb|Document|.)

\end{itemize}

\subsection*{Configuration}

You can configure exactly how this works using a \verb|DOMOutputOptions| Object,
which is a simple JavaBean that lets you control details such as:

\begin{itemize}
  \item Prefixing of XHTML, MathML (and SnuggleTeX) XML elements;
  \item Whether to annotate MathML with the LaTeX input;
  \item Whether and how to report LaTeX errors within the XML output;
  \item What to do with any URLs (e.g. those specified with the \verb|\href| command);
  \item Whether to inline CSS using \verb|style| attributes. (This can be useful if your XML
    is going to end up inside some kind of XML application that doesn't support user-specified
    CSS stylesheets. See \href[CSS Stylesheet Notes]{stylesheets.html} for more discussion of
    these points.)
  \item Whether you want to ``post-process'' the newly created DOM \verb|Node|s before they get
    added to the final DOM tree.
\end{itemize}

Have a look at the API documentation for \verb|DOMOutputOptions| for more information.
