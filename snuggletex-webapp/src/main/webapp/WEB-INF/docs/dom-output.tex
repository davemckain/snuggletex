\pageId{outputs}

\section*{Building a DOM}

SnuggleTeX can either append Nodes to an existing DOM, or return a DOM
\verb|NodeList| within a newly created \verb|Document|.

\subsection*{Usage}

\begin{itemize}
  \item
    Call \verb|snuggleSession.buildDOMSubtree(Element)|
    or \verb|snuggleSession.buildDOMSubtree(Element, DOMOutputOptions)|
    to append the resulting XML to an existing \verb|Element|'s children.

  \item
    Call \verb|snuggleSession.buildDOMSubtree()|
    or \verb|snuggleSession.buildDOMSubtree(DOMOutputOptions)|
    to create a new \verb|Document| and return a \verb|NodeList| representing
    your converted LaTeX inputs.
\end{itemize}

You can configure exactly how this works using a \verb|DOMOutputOptions| Object,
which is a simple JavaBean that lets you control aspects such as:

\begin{itemize}
  \item Prefixing of MathML elements;
  \item Whether to annotate MathML with the LaTeX input;
  \item Whether and how to report LaTeX errors within the XML output;
  \item How to remap any URLs discovered in documents (e.g. with the \verb|\href| command);
  \item Whether to inline CSS using \verb|style| attributes. (This can be useful if your XML
    is going to end up inside some kind of XML application that doesn't support user-specified
    CSS stylesheets.)
  \item Whether you want to ``post-process'' the newly created DOM \verb|Node|s before they
    added to the final tree.
\end{itemize}

Have a look at the API documentation for \verb|DOMOutputOptions| for more information.
