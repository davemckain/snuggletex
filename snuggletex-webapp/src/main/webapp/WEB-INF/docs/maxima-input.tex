\pageId{maxima}

\begin{itemize}
\item
  The Maxima input form is created from the Content MathML form. As a result,
  if the conversion to Content MathML fails then so will the conversion to
  Maxima.

\item
  Not all inputs that can be successfully converted to Content MathML can
  be further converted into Maxima input form.
\end{itemize}

\subsection*{Supported Identifiers}

MathML generally allows arbitrary Unicode characters to be used as identifier
names whereas Maxima only uses ASCII. In terms of LaTeX input, this means that
idenfiers input as \verb|\something| are only supported if we have identifier a
means to convert these into a Maxima input form.

The identifiers supported so far are:

\begin{tabular}{|c|c|}
\hline
LaTeX input & Maxima form \\
\hline
\verb|\alpha| & \%alpha \\
\verb|\beta| & \%beta \\
\verb|\gamma| & \%gamma \\
\verb|\delta| & \%delta \\
\verb|\epsilon| & \%epsilon \\
\verb|\zeta| & \%zeta \\
\verb|\eta| & \%eta \\
\verb|\theta| & \%theta \\
\verb|\iota| & \%iota \\
\verb|\kappa| & \%kappa \\
\verb|\lambda| & \%lambda \\
\verb|\mu| & \%mu \\
\verb|\nu| & \%nu \\
\verb|\xi| & \%xi \\
\verb|\pi| & \%pi \\
\verb|\rho| & \%rho \\
\verb|\sigma| & \%sigma \\
\verb|\tau| & \%tau \\
\verb|\upsilon| & \%upsilon \\
\verb|\phi| & \%phi \\
\verb|\chi| & \%chi \\
\verb|\psi| & \%psi \\
\verb|\omega| & \%omega \\
\verb|\Gamma| & \%Gamma \\
\verb|\Delta| & \%Delta \\
\verb|\Theta| & \%Theta \\
\verb|\Lambda| & \%Lambda \\
\verb|\Xi| & \%Xi \\
\verb|\Pi| & \%Pi \\
\verb|\Sigma| & \%Sigma \\
\verb|\Upsilon| & \%Upsilon \\
\verb|\Phi| & \%Phi \\
\verb|\Psi| & \%Psi \\
\verb|\Omega| & \%Omega \\
\hline
\hline
\end{tabular}

The LaTeX inputs \verb|e| and \verb|i| are converted to \%e
and \%i respectively if the Content MathML up-conversion
was configured to interpret them in this way.

\subsection*{Supported Functions}

All functions supported by the Content MathML up-conversion are supported
here, with the exception of \verb|\log| as Maxima only supports natural
logarithms.

\subsection*{Supported Operators}

All of the operators supported by the Content MathML process are
supported, with the exception of the following relations:

\begin{itemize}
  \item \verb|\equiv|
  \item \verb|\not\equiv|
  \item \verb|\approx|
  \item \verb|\not\approx|
  \item \verb.|.
  \item \verb.\not|.
  \item \verb|\in|
  \item \verb|\not\in|
\end{itemize}

Maxima does not allow ``unapplied'' operators symbols, so a LaTeX input
of the form \verb|+| generates a result of the form \verb|operator("+")|
as a placeholder for this. Similarly, \verb|\not=| results in
\verb|operator("not=")| for want of anything better.
The name of the resulting function is configurable.

\subsection*{Other Supported Constructs}

\begin{itemize}
\item
  Units entered using the special SnuggleTeX \verb|\units| macro, such as
  \verb|\units{kg}| generate a Maxima form like \verb|units("kg")|, which
  is similar to how we handle unapplied operators. (Again, the name of
  the resulting function is configurable.)

\item
  Subscripted identifiers are converted to a suitable Maxima form if
  deemed possible.
\end{itemize}

