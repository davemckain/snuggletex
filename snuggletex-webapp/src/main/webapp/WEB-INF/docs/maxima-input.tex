\pageId{maxima}

\newcommand{\ue}[1]{\upConversionExample{#1}}

SnuggleTeX can also attempt to convert input LaTeX to
\href[Maxima]{http://maxima.sourceforge.net/} input syntax.
This is implemented by first converting to Content MathML. Hence,
if the conversion to Content MathML fails then so will the conversion to
Maxima.

Also note that not all inputs that can be successfully converted to Content
MathML can be further converted into Maxima input form.

\subsection*{Supported Identifiers}

MathML generally allows arbitrary Unicode characters to be used as identifier
names whereas Maxima only safely supports ASCII characters. In terms of LaTeX
input, this means that identifiers input as \verb|\something| are only supported
if we have discovered a means to convert these into an appropriate Maxima input form.

The identifiers supported so far are:

\begin{tabular}{|c|c|c|}
\hline
LaTeX input & Maxima form & Example \\
\hline
\verb|\alpha| & \verb|\%alpha| & \ue{\verb|\alpha|} \\
\verb|\beta| & \verb|\%beta| & \ue{\verb|\beta|} \\
\verb|\gamma| & \verb|\%gamma| & \ue{\verb|\gamma|} \\
\verb|\delta| & \verb|\%delta| & \ue{\verb|\delta|} \\
\verb|\epsilon| & \verb|\%epsilon| & \ue{\verb|\epsilon|} \\
\verb|\zeta| & \verb|\%zeta| & \ue{\verb|\zeta|} \\
\verb|\eta| & \verb|\%eta| & \ue{\verb|\eta|} \\
\verb|\theta| & \verb|\%theta| & \ue{\verb|\theta|} \\
\verb|\iota| & \verb|\%iota| & \ue{\verb|\iota|} \\
\verb|\kappa| & \verb|\%kappa| & \ue{\verb|\kappa|} \\
\verb|\lambda| & \verb|\%lambda| & \ue{\verb|\lambda|} \\
\verb|\mu| & \verb|\%mu| & \ue{\verb|\mu|} \\
\verb|\nu| & \verb|\%nu| & \ue{\verb|\nu|} \\
\verb|\xi| & \verb|\%xi| & \ue{\verb|\xi|} \\
\verb|\pi| & \verb|\%pi| & \ue{\verb|\pi|} \\
\verb|\rho| & \verb|\%rho| & \ue{\verb|\rho|} \\
\verb|\sigma| & \verb|\%sigma| & \ue{\verb|\sigma|} \\
\verb|\tau| & \verb|\%tau| & \ue{\verb|\tau|} \\
\verb|\upsilon| & \verb|\%upsilon| & \ue{\verb|\upsilon|} \\
\verb|\phi| & \verb|\%phi| & \ue{\verb|\phi|} \\
\verb|\chi| & \verb|\%chi| & \ue{\verb|\chi|} \\
\verb|\psi| & \verb|\%psi| & \ue{\verb|\psi|} \\
\verb|\omega| & \verb|\%omega| & \ue{\verb|\omega|} \\
\verb|\Gamma| & \verb|\%Gamma| & \ue{\verb|\Gamma|} \\
\verb|\Delta| & \verb|\%Delta| & \ue{\verb|\Delta|} \\
\verb|\Theta| & \verb|\%Theta| & \ue{\verb|\Theta|} \\
\verb|\Lambda| & \verb|\%Lambda| & \ue{\verb|\Lambda|} \\
\verb|\Xi| & \verb|\%Xi| & \ue{\verb|\Xi|} \\
\verb|\Pi| & \verb|\%Pi| & \ue{\verb|\Pi|} \\
\verb|\Sigma| & \verb|\%Sigma| & \ue{\verb|\Sigma|} \\
\verb|\Upsilon| & \verb|\%Upsilon| & \ue{\verb|\Upsilon|} \\
\verb|\Phi| & \verb|\%Phi| & \ue{\verb|\Phi|} \\
\verb|\Psi| & \verb|\%Psi| & \ue{\verb|\Psi|} \\
\verb|\Omega| & \verb|\%Omega| & \ue{\verb|\Omega|} \\
\hline
\hline
\end{tabular}

The LaTeX inputs \verb|e| and \verb|i| are converted to the special
Maxima symbols \verb|\%e| and \verb|\%i| respectively if the Content MathML
up-conversion was configured to interpret them in this way.

\subsection*{Supported Functions}

All functions supported by the Content MathML up-conversion are supported
here, with the exception of \verb|\log| as Maxima only supports natural
logarithms.

\subsection*{Supported Operators}

All of the operators supported by the Content MathML process are
supported, with the \textbf{exception} of the following relations:

\begin{itemize}
  \item \verb|\equiv|
  \item \verb|\not\equiv|
  \item \verb|\approx|
  \item \verb|\not\approx|
  \item \verb.|.
  \item \verb.\not|.
  \item \verb|\in|
  \item \verb|\not\in|
\end{itemize}

Maxima does not support ``unapplied'' operators symbols, so a LaTeX input
of the form \verb|+| generates a result of the form \verb|operator("+")|
as a rather ad hoc representation of this idea. Similarly, \verb|\not<| results
in \verb|operator("not<")| for want of anything better.
The name of the resulting ``function'' is configurable.

\ue{\verb|+|}
\ue{\verb|\not=|}
\ue{\verb|\not<|}

\subsection*{Other Supported Constructs}

\begin{itemize}
\item
  Units entered using the special SnuggleTeX \verb|\units| macro, such as
  \verb|\units{kg}| generate a Maxima form like \verb|units("kg")|, which
  is similar to how we handle unapplied operators. (Again, the name of
  the resulting ``function'' is configurable.)

\item
  Subscripted identifiers are converted to a suitable Maxima form if
  deemed possible.

  \ue{\verb|A_{i,j}x_j|}
  \ue{\verb|x_{i,j_{n,m_2}}|}

\end{itemize}

