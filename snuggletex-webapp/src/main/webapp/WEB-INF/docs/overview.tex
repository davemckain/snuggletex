\pageId{overview}

SnuggleTeX is a free and open-source Java library for converting fragments
of LaTeX to XML (usually XHTML + MathML).

\begin{itemize}

\item
  SnuggleTeX converts math mode LaTeX to MathML, generating Presentation
  MathML by default.

\item
  As of SnuggleTeX 1.1.0, SnuggleTeX can also attempt to
  ``\href[semantically enrich]{docs://upconversion}'' the Presentation MathML it
  creates and generate Content MathML and \href[Maxima]{http://maxima.sourceforge.net/}
  input formats if it can make sense of its LaTeX input.  (This is aimed primarily at
  secondary/early tertiary UK Education contexts.)

\item
  SnuggleTeX converts text mode LaTeX to XHTML.

\item
  SnuggleTeX converts standard LaTeX inputs containing a mixture of text and math mode
  LaTeX to XHTML with embedded MathML.

\item
  SnuggleTeX can output either a DOM fragment, an XML fragment or create a full
  standalone web page.

\item
  SnuggleTeX has optional features for converting the resulting MathML to images
  (using the JEuclid library) and attempting to convert simple MathML expressions
  into a mixture of XHTML and CSS.

\item
  Web page outputs can be configured in various ways and SnuggleTeX provides
  a number of useful web page templates that can be used to target certain browsers
  and rendering options.

\item
  Outputs can be fully standalone (e.g. with all CSS styling done inline) to enable
  XML fragments to be imported into other XML applications and/or systems
  which support XHTML+MathML.

\item
  Error reporting is configurable and error messages are internationalisable and
  provide detailed contextual information.

\item
  SnuggleTeX supports \verb|\newcommand| and \verb|\newenvironment| and
  friends, making it easy to create custom commands and environments. There are
  also lower-level Java hooks for defining new commands and environments.

\item
  SnuggleTeX (1.1.0 and above) includes some experimental utility code to convert the raw MathML
  generated by \href[ASCIIMathML]{http://www1.chapman.edu/~jipsen/asciimath.html}
  to Content MathML and Maxima notation.

\end{itemize}
