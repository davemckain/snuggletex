\pageId{usageoverview}

\section*{Usage Overview}

\subsection*{The \verb|SnuggleTeXEngine|}

To use SnuggleTeX, you need to create an instance of \verb|SnuggleTeXEngine|.

\begin{verbatim}SnuggleTeXEngine engine = new SnuggleTeXEngine();\end{verbatim}

An instance of a \verb|SnuggleTeXEngine| is thread-safe and can be used
to perform as many conversions as you like. It houses a number of defaults,
options, command and environment definitions that can be set if you want to
perform a number of very similar conversion processes (which is quite common).

Despite its heavy-weight name, creating and setting up a \verb|SnuggleTeXEngine|
is not expensive!

\subsection*{The \verb|SnuggleTeXSession|}

A \verb|SnuggleTeXSession| encapsulates a single SnuggleTeX conversion "job",
which typically consists of the following steps:

\begin{enumerate}
  \item Parse one (or more) \href[\verb|SnuggleInput|s]{inputs.html}.
  \item (Check for any \href[errors]{error-reporting.html} in the LaTeX\ldots)
  \item Generate one (or more) \href[Outputs]{outputs.html}.
\end{enumerate}

To create a new \verb|SnuggleTeXSession|, you can use:

\begin{verbatim}SnuggleTeXSession session = engine.createSession();\end{verbatim}

This creates a fresh job using the current default settings in your
\verb|SnuggleTeXSession|. You can then call methods described in
\href[Inputs]{inputs.html}, \href[Error Reporting]{error-reporting.html}
and \href[Outputs]{outputs.html} to do whatever it is you need to do.

See the \href[Minimal Example]{minimal-example.html} for a simple example
of how everything comes together.

You can discard a \verb|SnuggleTeXSession| once it has done everything
you need. (It is also possible to create a \verb|Snapshot| of the state
of a \verb|SnuggleTeXSession|, allowing you to recreate a fresh
session having the same state later on. This is very useful if you want
to reuse inputs containing definitions of commands and environments.)
Note that a \verb|SnuggleTeXSession| is \emph{not} thread safe.
