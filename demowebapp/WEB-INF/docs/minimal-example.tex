\pageId{minexample}

\section*{Minimal Example}

Look at the \verb|MinimalExample.java| file in the full binary distribution.
This demonstrates a very simple example of calling up SnuggleTeX:

\begin{verbatim}
SnuggleTeXEngine engine = new SnuggleTeXEngine();
SnuggleTeXSession session = engine.createSession();

SnuggleInput input = new SnuggleInput("$$1+2=3$$");
session.parseInput(input);
String xmlString = session.buildXMLString();

System.out.println("Input " + input.getString()
    + " was converted to:\n" + xmlString);
\end{verbatim}

This block of code converts the LaTeX String \verb|$$1+2=3$$| to a simple XML
String fragment, which is printed to the standard output.

For this particular input, the resulting XML fragment consists of a single MathML
\verb|math| element.

The XML fragment output in the above example is great for showing off the outputs
from SnuggleTeX but is not usually much use if you want to do anything else with it.
SnuggleTeX supports a number of alterative \href[outputs]{outputs.xml} that can be
more useful to programmers.
